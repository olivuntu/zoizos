\documentclass{article}
\usepackage{graphicx}
\usepackage{amssymb,amsfonts,amsmath}
\usepackage[natbib=true, bibstyle=authoryear, citestyle=authoryear-comp]{biblatex}
\usepackage[T1]{fontenc}
\usepackage[utf8]{inputenc}
\usepackage{aeguill}
\usepackage{inputenc}
\usepackage{fullpage}

\newcommand\R{\rule{0pt}{2.5ex}}
\newcommand\B{\rule{0pt}{-1ex}}
\renewcommand{\belowcaptionskip}{1ex}

\setlength{\parindent}{0cm}

\bibliography{/home/olive/Boulot/biblio/db/biblio}

%%%%%%%%%%%%%%%%%%%%%%%%%%%%%%%%%%%%%%%%%%%%%%%%
\title{Elevational distribution patterns differ between exotic and native birds in Reunion Island}
%\author{Olivier Flores}

%%%%%%%%%%%%%%%%%%%%%%%%%%%%%%%%%%%%%%%%%%%%%%%%

\begin{document}
<<opts, echo=F, cache=F>>=
render_latex()
opts_knit$set(root.dir='/home/olive/Boulot/zoizos',base.dir='docs')
opts_chunk$set(fig.path='figures/zoizos-',cache=T,cache.path='docs/cache/',fig.align='center',fig.show='hold',fig.width=7,fig.height=7,fig.ext='jpg')
read_chunk('../code/zoizos_figs.r')
@ 


%<<echo=F,label=data,cache=true,include=F>>=
%chunk="data"
%source("../R/data_zoizo.r")
%@ 

\maketitle

\noindent

\begin{abstract}
%Although biological invasions are one of the most important changes in ecological dynamic, they have been rarely investigated along elevational gradients. The objective of this study was to compare the distributions of exotic versus native birds on a same broad elevational gradient. Field studies were conducted on the leeward side of Reunion Island (Mascarene Archipelago, Western Indian Ocean) between 20 and 2880 m at 363 sampling points. Species richness and abundances of bird species were surveyed during the nesting season using 20 min count at each point. Mean elevational position and elevational amplitude were measured for each species and a Mean Distribution Index was computed as the number of bird species having their mean elevation within each 100 m elevational band. The data show that (1) the elevational variation of native species richness is hump-shaped, whereas the richness of exotic bird species does not vary in the first half of the gradient; (2) the elevational amplitude and maximum elevation do not differ significantly between exotic and native birds, but the mean and minimum elevations are higher for native birds; and (3) the most suitable vegetation types are found in the ‘rural landscapes’ belt for the exotic bird communities, and in the higher belt of ‘forest and pastures’ for the native ones. Even if history can appear too short to permit niche expansion of the more recent introduced species, we can underline the exotic species preference for anthropogenic landscapes and their large amplitude in elevation. 
\end{abstract}


%%%%%%%%%%%%%%%%%%%%%%%%%%%%%%%%%%%%%%%%%%%%
\section{Introduction}

%The conspicuous ecological changes that occur along elevational gradients have drawn the attention of an increasing number of studies. Most hypotheses which attempted to explain biological responses to elevational gradients were formally based on a single bio-physical factor such as productivity, habitat diversity, environmental stress, resource availability or competition (e.g. Rahbek, 1995; Heaney, 2001). Others underlined the role of anthropogenic pressure or ecological and evolutionary processes (e.g. Myers \& Giller, 1988). 
%Surprisingly, studies on species distribution along elevational gradients scarcely pay attention to the status of the involved species, i.e. either exotic or native, despite of the increased importance of the impact of introduced species on biodiversity (Vitousek et al., 1997). The same can be said of studies on latitudinal gradients (but see Sax, 2001). Considering the theoretical framework of biological invasions, investigations on the distribution of exotic species along broad ecological gradients should enable us to predict invasion patterns. 


%Birds, widely introduced throughout the world and having colonized most vegetation types (Long, 1981), are a relevant taxon for studying the comparative distribution of exotic versus native species along elevational gradients. Exotics are species that have been introduced by humans, or have been able to expand their range because of anthropogenic disturbances, into geographical regions in which they were not historically present (Sax, 2001). Birds have been widely used as models in studies on the variation of species richness with elevation (Thiollay, 1980; Patterson et al., 1998; Benning et al., 2002 ; Prodon et al., 2002). The elevational range of bird communities is also of special interest as it is one of the dimensions of the niche of a species, and it is linked to flexibility in habitat or food choices, ability to coexist with different species, and physiological tolerance (Prodon et al., 2002). % Last, the mean elevational position can be considered to be an indicator for determining one of the optimal vegetation types of a species. 


In this paper, we use past data to document the elevational patterns of bird assemblages on a tropical volcanic island.  We specifically adress the question whether elevational distribution patterns differ between exotic and native bird species. We analyse patterns along a slope on Reunion Island (Indian Ocean) that presents one of the widest bio-climatic gradients within any oceanic island, \textit{i.e.} from 0 to more than 3000 m on a distance of 20 km. We first characterize species distribution along the gradient and compare distribution parameters across native and exotic species. We then characterize patterns of species richness and compare the observed patterns to two different nul models. 

Analysing observed patterns relatively to nul models expectations can inform on assembly processes. Nul models differ in the way they assemble community data, that is species occurrence data, here abundance, in sampled sites. They can include several constraints which consist in preserving some properties of the observed data, such as species overall abundance or local abundance at each sampled site. Here, the two tested models assemble species randomly preserving their frequency along the gradient, in order to retain differences in rarity. The first model is constrained by the total abundance of birds observed locally that is held constant during estimation. The model thus conserves interactions between total abundance and the environment, which can be referred to as the 'site capacity hypothesis'. 


This capacity is supposed to be mobilized by birds present to some constant level, which is reflected in the total estimated abundance. 

In the second model, these interactions are suppressed supposing that site capacity varies randomly. In this model, the sites are supposed as equal in their capacity to support birds to some random level. 

Comparing observed data with nul models allows to explore differences in the elevational patterns of species richness between native and exotic birds under the constraint of observed commonness and site capacity or not. 



Studies of this type of oceanic islands may contribute to understand the potentially huge impact of biological invasions (Loope \& Mueller-Dombois, 1989). 



%Nineteen native bird species on the island among which five are classified as endangered (Coracina newtoni, Pseudobulweria aterrima, Circus
%maillardi, Pterodroma baraui, Psittacula echo. Twenty exotic species have established populations on the island. 

%%%%%%%%%%%%%%%%%%%%%%%%%%%%%%%%%%%
\section{Material and methods}

\subsection{Studied area}
Reunion Island is part of the Mascarene volcanic archipelago, situated in the Western Indian Ocean (55°30’ E and 21°05’ S)(Fig. 1). This oceanic island appeared three million years ago. In the centre of the island, a plateau (Plaine des Caffres, $\approx$2000 m) is flanked by two volcanoes (Piton de la Fournaise, 2631 m; Piton des Neiges, 3069 m), the sides of which slope steeply towards the sea. There is a very steep climatic gradient: the average minimum ground temperature falls below 0°C above 1500 m in August, above 1800 m from June to October, and above 2300 m all year round; at the highest point, the temperature ranges from -10°C to +26°C \citep{Cadet1974,Raunet1991}.
The studied area, located on the leeward side of the island had a slope of about 0.15 m per m and an annual rainfall ranging from 500 mm at sea level to 2000 mm on the highest mountain slopes \cite{Raunet1991}. The land is roughly organized through a land use gradient into a succession of elevational belts \citep{Cadet1980} urbanised areas (0-50 m a.s.l.), herbaceous savannah (50-300 m), sugarcane monoculture (300-800 m), rural landscapes with diversified crops (800-1200 m), primary forest mixed with pastures and planted forests (1200-1800 m), heathlands dominated by ericaceous plants (1800-2600 m), and bare soils with a very sparse vegetation of herbaceous species (2600-3069 m) (Fig. ??). Above 1200 m, about 60 % of the original primary forest and 80 % of the original heathlands have been preserved from clearing (Strasberg et al., 2006).

\subsection{Sampling}


The transect was sampled through one survey during the breeding season (December, 1997 - April, 1998), using 363 sample points spread along an elevational gradient from 20 m to 2880 m (Fig. ??). Further prospecting between 2880 and 3069 m (Piton des Neiges) revealed no bird presence. Sea birds nest at these high altitudes but were not considered in the study. This was due to of a lack of vegetation, except for scarce amounts of native herbs and weeds. Sampling points were equally distributed all along the elevational gradient, with a mean elevational distance of 8 m between two successive elevations, and a distance of at least 250 m between two successive count points, a mean sampling pressure of 12.7 count points per 100 m elevational band, and covering the overall diversity of vegetation types. Depressions, ridges and mires were avoided in order to avoid extremely wet or dry places. We also avoided places too close to feeding troughs in grasslands as they attract granivorous bird species. We assume that most of the birds were sampled near their nesting sites. In order to limit sampling bias, the different ecosystems found at each 100~m elevational band were surveyed so as not to miss individual bird species specific to individual ecosystems. 

%An equal sampling effort may still result in sampling unequal proportions of the total richness, which may in turn affect the elevational species richness pattern once interpolation is used \citep{Colwell1994}. One reason for this is that, in this case, richness towards the boundaries only consists of observed species, whereas at other elevations richness consists of observed species plus the species added by interpolation (Grytnes \& Vetaas, 2002)



\subsection{Data analysis}

%\subsection{Analysis}

%Elevation patterns of species composition of exotic versus native bird communities were based on elevational bands of 100 m investigated in three different ways: (1) the variation of species richness with elevation was analyzed, (2) the distribution of species along the elevational gradient was characterized with regards to mean elevation and range, and (3) the current optimal elevational band of each species in terms of reproductive success and fitness. Most studies on elevational gradients focus on the variation of species richness with elevation but do not give any information on the mean elevational position, that is to say the optimal elevational position of the studied taxa (Rahbek, 1997; Paterson et al., 1998; Odland \& Birks, 1999; Kessler, 2000; Grytnes, 2003). In terms of conservation, it appears sound to identify those elevational bands that are optimal for a great number of species, even if we must consider that some species have been displaced from their optimal location by human land use and conversely that human land use has added resources to the landscape. The investigation of the elevational distribution of the mean elevations of studied species yields information that the mere examination of variation of species richness with elevation will not provide. Therefore, for each 100m elevational band, we computed a Mean Distribution Index MDI which is the number of bird species having their mean elevation in band. Only species recorded in more than five point counts were used in the computation.
%Comparisons between exotic and native species were performed using the Man-Whitney test. Correlations were computed using Pearson product-moment correlation coefficients and tested using the Bonferroni adjusted probability. All statistical analyses were computed using R software.


We calculated the weighted mid-point of species distribution as the average of all elevations where a species was detected weighted by the number of times it was detected. We also compare species maximal and minimal elevation where observed along the gradient, and amplitude in elevation range. We use the word elevation instead of altitude, as elevation is meters above sea level where altitude is meters above ground.

We compared observed species richness in the two groups of bird species (native and exotic) to two different nul models based on random permutations of the original dataset  \cite{Gotelli2000}. In both models, species overall abundance, that is the total number of records for each species, was preserved in each permutation, in order to account for observed differences in species commonness and detectability. The main difference between the two nul models was in the treatment of site carrying capacity of birds diversity, which under equilibrium hypothesis can be estimated as the total numbers of individuals locally. In the first nul model (N1), sites are considered as equally suitable for birds. In this case, the randomization suppresses any structure that exists in the data due to relationships between carying capacity and environment (here elevation) in a broad sense. The randomized pattern obtained mimics the distribution of species richness not accounting for those relationships. In the second nul model (N2), the carying capacity of sites, that is local species richness, was preserved by the randomization procedure. This model holds account for differences in sites to support bird abundance in relation with habitat carrying capacity. In this case, the randomized pattern obtained takes into account both species commonness and site suitability. The total number of individuals observed locally is preserved and species abundance in the dataset are preserved, but individuals and samples are randomized across sites. For each model, we performed a number of permutations ($n=1999$) to obtain 95-\% confidence intervals on simulated species richness. 


%%%%%%%%%%%%%%%%%%%%%%%%%%%%%%%%%%%
\section{Results}

% General numbers
Twenty-two bird species were recorded over the 363 sampling points, with a mean of \Sexpr{round(mean(rowSums(zspp!=0)),digits=1)} species per point and \Sexpr{round(mean(Se+Sn),digits=1)} species per 100-m elevational band along the gradient.  Fourteen species were exotic and eight native (Table \ref{species}). The number of records varied widely across species between 1 for two species (\textit{\Sexpr{ztax$Species[ztax$code=="COCH"]}, \Sexpr{ztax$Species[ztax$code=="PHBO"]}}) and 268 for \textit{\Sexpr{ztax$Species[ztax$code=="ZOBO"]}} (mean: \Sexpr{round(mean(colSums(zspp!=0)),digits=1)} $\pm$ s.d.: \Sexpr{round(sd(colSums(zspp!=0)),digits=1)}, Fig. \ref{boxalti}). 

% Distribution parameters
Amplitudes in elevational range were high. Elevational amplitude exceeded 2000~m for five species and 1000~m for 15 species (Table 1). The largest elevational amplitude measured in native species was for \textit{Zosterops borbonicus} (92 \% of the gradient) and for \textit{Acridotheres tristis} in exotic species (83 \%) . The lowest elevational amplitude measured in native species was registered for \textit{Circus maillardi} (20 \%) and for \textit{Perdicula asiatica} in exotic species (8 \%). The distribution of weighted mid-points, that is the weighted mean of elevations where a species occurs along the gradient, was highly concentrated between 1000 and 1500 m, with the exception of three exotic species (Fig. \ref{boxrange}). The weighted mid-point and minimum elevation were however higher in native species ($p=0.034$ and $p=0.024$). By contrast, elevational amplitude and maximum elevation did not differ significantly between the two groups of species ($p=0.534$ and $p=0.254$ respectively, Fig. \ref{boxrange}). No difference in the variability of species distribution was observed across the two groups: Bartlett test's of variance homogeneity showed that the variability was similar in the two groups for the four tested variables.

 % Patterns in species richness
The total species richness observed within 100-m elevation bands varied between 2 and 19 (10.4$\pm$5.5, Fig. \ref{Sgam}.a). Total richness showed a unimodal mid-elevational peak between 1000 and 1200~m. However, the pattern of species richness differed significantly between exotic and native species with increasing elevation (Fig. \ref{Sgam}). Species richness varied between 0 and 7 for native birds (3.5$\pm$2.2), and between 1 and 12 (6.9$\pm$3.9) for exotic birds (Fig. \ref{Sgam}.b). Species richness showed different elevational patterns across the two groups. Species richness of native birds showed a unimodal mid-elevational peak around 1200~m, whereas richness for exotic birds species showed a low-elevation plateau with a small mid-peak also around 1200~m. Exotic species were more numerous than native birds up to mid-elevation, whereas native species were more numerous at higher altitudes (Fig. \ref{Sgam}.b). According to richness patterns, three zones could be defined: (1) from 0 to 900 m, there was a significant difference between exotic and native species, both in the number of species contacted and in their variation pattern, as native species increased with the elevational gradient from two to seven while the exotic ones remained at a high level (around 9 species ); (2) from 1000 to 2000 m, richness of both exotic and native species presented a peak (around 9 and 6 species, respectively); (3) from 2000 to 3000~m, richness was higher in native species while exotic species were barely present. 

% Null models: comparison
In order to explore species co-occurence according to their status, we compared species richness to random patterns obtained by simulation. Values occurring within the simulated 95\% confidence band for the random pattern indicated elevation bands where species richness was not statistically different from random patterns of species assemblage (Fig. \ref{Snul}). In both groups, null patterns showed higher variability at low and high elevation when sites were all considered as equally suitable (randomized total abundance), whereas null patterns were more constrained when the total abundance observed at each site was kept constant (site capacity hypothesis). All models showed mid-elevation peaks however less pronounced under the site capacity hypothesis.

% Comparison to null patterns by group
Native species richness was significantly lower than random expectations at low elevation for both null models (Fig. \ref{Snul}.a). This finding indicates depauperate communities where native species occur less frequently than expected by chance.  From 900~m above, the observed pattern in native species richness overall supported more the site capacity hypothesis Regarding exotic birds, species richness was lower than expected by chance above 1500~m high (Fig. \ref{Snul}.b). This result indicates that exotic species in high elevation bird communities are in low numbers compared to the site capacity. At low to mid altitude, the observed pattern in exotic species richness was closer to the confidence interval of the null model without constraint on site capacity (equally suitable hypothesis). The site capacity hypothesis generated higher than observed numbers of exotic species. 


%%%%%%%%%%%%%%%%%%%%%%%%%%%%%%%%%%%
\section{Discussion}

%\printbibliography
\section{Tables}

% latex.default(era, cdec = c(0, rep(0, 5)), caption = "Summary of species elevation patterns",      ctable = FALSE, label = "era", file = "tables/zoizos_era.tex") 
%
\begin{table}[!h]
\caption{Summary of species elevation patterns. Occurrence: number of times the species was recorded. Mean : average elevation where the species was observed. Weighted mean: average altitude weighted by the number of times the species was recorded at each site. Min., Max.; minimum and maximum elevation at which the species was observed.\label{era}} 
\begin{center}
\begin{tabular}{lccccc}
\hline
\multicolumn{1}{l}{\R Species}&\multicolumn{1}{c}{Occurrence}&\multicolumn{1}{c}{Mean (m)}&\multicolumn{1}{c}{Weighted mean (m)}&\multicolumn{1}{c}{Min.}&\multicolumn{1}{c}{Max.}\tabularnewline
\hline
\R ACTR&$168$&$1084$&$1123$&$  20$&$2400$\tabularnewline
CIMA&$  6$&$1058$&$1049$&$ 800$&$1370$\tabularnewline
COCH&$  1$&$1010$&$1010$&$1010$&$1010$\tabularnewline
COCO&$ 33$&$1267$&$1307$&$ 320$&$1700$\tabularnewline
COFR&$ 57$&$1115$&$ 974$&$ 185$&$1660$\tabularnewline
ESAS&$ 88$&$1001$&$ 987$&$  60$&$1720$\tabularnewline
FOMA&$268$&$1051$&$1033$&$  20$&$2250$\tabularnewline
FRPO&$  2$&$ 128$&$ 142$&$  70$&$ 185$\tabularnewline
GEST&$109$&$ 840$&$ 804$&$  20$&$1620$\tabularnewline
HYBO&$ 67$&$1405$&$1410$&$ 620$&$2560$\tabularnewline
MAMA&$  5$&$1528$&$1615$&$1240$&$1705$\tabularnewline
PADO&$ 72$&$1128$&$1202$&$  65$&$1960$\tabularnewline
PEAS&$ 16$&$ 117$&$ 119$&$  50$&$ 290$\tabularnewline
PHBO&$  1$&$1495$&$1495$&$1495$&$1495$\tabularnewline
PLCU&$ 32$&$ 580$&$ 606$&$  65$&$1345$\tabularnewline
PYJO&$ 72$&$ 794$&$ 805$&$  65$&$1690$\tabularnewline
SATE&$202$&$1590$&$1601$&$ 270$&$2880$\tabularnewline
STPI&$ 34$&$1181$&$1225$&$ 100$&$2150$\tabularnewline
TEBO&$ 18$&$1307$&$1336$&$ 900$&$1680$\tabularnewline
TUNI&$ 41$&$ 832$&$ 808$&$  35$&$1700$\tabularnewline
ZOBO&$229$&$1371$&$1347$&$  60$&$2700$\tabularnewline
ZOOL&$ 45$&$1391$&$1388$&$ 620$&$2600$\tabularnewline
\hline
\end{tabular}
\end{center}
\end{table}



%%%%%%%%%%%%%%%%%%%%%%%%%%%%%%%%%%%
\section{Figure legends}

Figure \ref{boxalti}: Box-and-whisker plots indicating the median, first and third quartiles and values outside inter-quartile range of elevations where species were detected. Species are sorted by order of median evelation. Blue indicates native species, , and red exindicate exotic species. Species are labelled with the first two letters of the genus and species names (see Table \ref{species}).

Figure \ref{boxrange}: Four features of species distributions with respect to elevation in native and exotic species : (a) weigthed mid elevation point (average elevation where the species was detected weighted by number occurences), (b) elevation amplitude (difference between highest and lowest elevation of detection), (c) minimum elevation where the species was detected and (d) maximal elevation of detection. Box-and-whisker plots indicate the median, first and third quartiles and values outside inter-quartile range. Symbols indicate individual values ($n=22$ species), open symbols indicate species with less than 5 records in dataset ($n=3$ species, see Table \ref{era}).

Figure \ref{Sgam}: Elevation patterns of species richness : (a) total and (b) according to species status, native or exotic. Richness is calculated here per 100-m elevation band as the number of species detected within the band. Dotted lines indicate generalized additive models fits (GAM) with colored bands showing the 95\% confidence interval on estimated richness. 

Figure \ref{Snul}: Comparison between observed richness (dots) within 100-m elevation bands for native (a. \& c.) and exotic (b. \& d.) birds and expected species richness under two different null models that simulate random species assemblages (95\%-confidence band). Green lines:  null model with fixed site capcity, blue lines: null model with random site capacity (equally suitable sites hypothesis). In both models, species frequencies in the dataset were preserved during simulation ($n=2000$).

%%%%%%%%%%%%%%%%%%%%%%%%%%%%%%%%%%%
\section{Figures}

% % % % % % % % % % % % % % % % % % % % % % % % % %
	
\begin{figure}
\centering
%<<boxes,echo=F,cache=T>>=
%@ 
\begin{tabular}{cc}
(a) & (b) \\
<<boxwmalt,echo=F,cache=T,out.width='0.5\\textwidth'>>=
@ 
&
<<boxampl,echo=F,cache=T,out.width='0.5\\textwidth'>>=
@ 
\\
(c) & (d) \\
<<boxmin,echo=F,cache=T,out.width='0.5\\textwidth'>>=
@ 
&
<<boxmax,echo=F,cache=T,out.width='0.5\\textwidth'>>=
@ 
\end{tabular}
\caption{\label{boxrange}}
\end{figure}


% % SPECIES RANGE % % % % % % % % % % % % % % % % % % % %
\begin{figure}
<<boxalti,echo=F,cache=T,include=T>>=
@
\caption{\label{boxalti}}
\end{figure}



% % % % % % % % % % % % % % % % % % % % % % % % % % % % % % %
% % GAM FITS
\clearpage

\begin{figure}
\hspace{-2cm}
\begin{tabular}{cc}
a& b\\
<<Stot,echo=F,cache=F,out.width='0.5\\linewidth'>>=
@ 
&
<<Snatexo,echo=F,cache=F,out.width='0.5\\linewidth'>>=
@ 
\end{tabular}
\caption{\label{Sgam}}
\end{figure}

\clearpage

\begin{figure}
\hspace{-2cm}
\begin{tabular}{cc}
<<Snatnul,echo=F,cache=T,out.width='0.5\\linewidth'>>=
@ 
&
% % NULL MODEL COLUMN MARGIN PRESERVED
<<Sexonul,echo=F,cache=T,out.width='0.5\\linewidth'>>=
@ 
\end{tabular}
\caption{\label{Snul}}
\end{figure}

% % % % % % % % % % % % % % % % % % % % % % % % % %
\clearpage
\section{Supporting information}
\renewcommand{\thefigure}{S\arabic{figure}}
\setcounter{figure}{0}

\renewcommand{\thetable}{S\arabic{table}}
\setcounter{table}{0}
\begin{table}
\caption{List of bird species detected in the study\label{tbl:species}}
\footnotesize
\begin{tabular}{|llllcccc|}
\toprule
\multicolumn{1}{|c}{\textbf{Label}}& \multicolumn{1}{c}{\textbf{Species}} & \multicolumn{1}{c}{\textbf{Order}} & \multicolumn{1}{c}{\textbf{Family}} & \textbf{Mass} (g) & \textbf{Size} (cm) & \textbf{Status} & \textbf{RLI} \\
\midrule
ACTR & Acridotheres tristis & Passeriformes & Sturnidae & 82 – 134 & 23 – 29 & E & LC \\ 
CIMA & Circus maillardi & Accipitriformes & Accipitridae & 50 & 650 – 1000 & R  & EN  \\
COFR & Collocalia francica & Apodiformes & Apodidae & 7 – 13,3 &  & M & NT \\
COCH & Coturnix chinensis & Galliformes & Phasianidae & 31 – 41 & 12 – 15 & E & LC \\
COCO & Coturnix coturnix & Galliformes & Phasianidae & 16 & 70 - 135 & E & LC \\ 
ESAS & Estrilda astrild & Passeriformes & Estrildidae & 7 – 8 & 10 & E & LC \\ 
FOMA & Foudia madagascariensis & Passeriformes & Ploceidae & 13 & 17 – 19 & E & LC \\ 
FRPO & Francolinus pondicerianus & Galliformes & Phasianidae & 200 - 340 & 30 - 32 & E & LC \\ 
GEST & Geopelia striata & Columbiformes & Columbidae &  & 20 & E & LC \\ 
HYBO & Hypsipetes borbonicus & Passeriformes & Pycnonotidae & 22 & 51 - 57 & R  & LC \\ 
%LOPU & Lonchura punctata & Passeriformes & Estrildidae & 11 & 12 – 15 & E & LC \\ 
MAMA & Margaroperdrix madagascariensis & Galliformes & Phasianidae &  & 24 - 26 & N & LC \\ 
PADO & Passer domesticus & Passeriformes & Passeridae & 26 - 32 & 14 - 16 & E & LC \\ 
PEAS & Perdicula asiatica & Galliformes & Phasianidae & 57 - 82 & 17 & N & LC \\ 
PHBO & Phedina borbonica & Passeriformes & Hirundinidae &  & 13,5 & E & LC \\ 
PLCU & Ploceus cucullatus & Passeriformes & Ploceidae & 17 & 32 – 45 & E & LC \\ 
PYJO & Pycnonotus jocosus & Passeriformes & Pycnonotidae & 20 & 23 – 42 & E & LC \\ 
SATE & Saxicola tectes & Passeriformes & Muscicapidae & 12,5 & 10,5 – 15 & R  & LC \\ 
STPI & Streptopelia picturata & Columbiformes & Columbidae & 28 & 145 – 255 & N & LC  \\ 
TEBO & Terpsiphone bourbonnensis bourb. & Passeriformes & Monarchidae & 15 & 10 – 12 & M (R)  & LC \\ 
TUNI & Turnix nigricollis & Turniciformes & Turnicidae &  & 13 & N & LC  \\ 
ZOBO & Zosterops borbonicus borbonicus & Passeriformes & Zosteropidae & 10 & 7 – 8 & M (R)  & LC \\ 
ZOOL & Zosterops olivaceus & Passeriformes & Zosteropidae & 10,5 & 9 – 10 & R  & LC \\ 
\bottomrule
\end{tabular}
\end{table}

\end{document}}
